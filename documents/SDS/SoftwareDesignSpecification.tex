% This file was converted to LaTeX by Writer2LaTeX ver. 1.0.2
% see http://writer2latex.sourceforge.net for more info
\documentclass[twoside,letterpaper]{article}
\usepackage[latin1]{inputenc}
\usepackage[T1]{fontenc}
\usepackage[english]{babel}
\usepackage{amsmath}
\usepackage{amssymb,amsfonts,textcomp}
\usepackage{color}
\usepackage{array}
\usepackage{supertabular}
\usepackage{hhline}
\usepackage{hyperref}
\usepackage{datetime}
\hypersetup{pdftex, colorlinks=true, linkcolor=blue, citecolor=blue, filecolor=blue, urlcolor=blue, pdftitle=SYSTEM AND SOFTWARE ARCHITECTURAL AND DETAILED DESIGN DESCRIPTI, pdfauthor=Clinton Jeffery, pdfsubject=, pdfkeywords=}
\usepackage[pdftex]{graphicx}
% Outline numbering
\setcounter{secnumdepth}{5}
\renewcommand\thesection{\arabic{section}}
\renewcommand\thesubsection{\arabic{section}.\arabic{subsection}}
\renewcommand\thesubsubsection{\arabic{section}.\arabic{subsection}.\arabic{subsubsection}}
\renewcommand\theparagraph{\arabic{section}.\arabic{subsection}.\arabic{subsubsection}.\arabic{paragraph}}
\renewcommand\thesubparagraph{\arabic{section}.\arabic{subsection}.\arabic{subsubsection}.\arabic{paragraph}.\arabic{subparagraph}}
\makeatletter
\newcommand\arraybslash{\let\\\@arraycr}
\makeatother
% List styles
\newcommand\liststyleWWviiiNumii{%
\renewcommand\theenumi{\arabic{enumi}}
\renewcommand\theenumii{\arabic{enumii}}
\renewcommand\theenumiii{\arabic{enumiii}}
\renewcommand\theenumiv{\arabic{enumiv}}
\renewcommand\labelenumi{\theenumi)}
\renewcommand\labelenumii{\theenumii.}
\renewcommand\labelenumiii{\theenumiii.}
\renewcommand\labelenumiv{\theenumiv.}
}
% Page layout (geometry)
\setlength\voffset{-1in}
\setlength\hoffset{-1in}
\setlength\topmargin{0.5in}
\setlength\oddsidemargin{1in}
\setlength\evensidemargin{1in}
\setlength\textheight{8.278in}
\setlength\textwidth{6.5in}
\setlength\footskip{0.561in}
\setlength\headheight{0.5in}
\setlength\headsep{0.461in}
% Footnote rule
\setlength{\skip\footins}{0.0469in}
\renewcommand\footnoterule{\vspace*{-0.0071in}\setlength\leftskip{0pt}\setlength\rightskip{0pt plus 1fil}\noindent\textcolor{black}{\rule{0.25\columnwidth}{0.0071in}}\vspace*{0.0398in}}
% Pages styles
\makeatletter
\newcommand\ps@Standard{
  \renewcommand\@oddhead{\selectlanguage{english}\rmfamily\color{black} Sabanc{\i} University CS Department Instructional Use\hfill \hfill Scheduler}
  \renewcommand\@evenhead{\@oddhead}
  \renewcommand\@oddfoot{\foreignlanguage{english}{\textcolor{black}{\centerline{\thepage{}}}}}
  \renewcommand\@evenfoot{\@oddfoot}
  \renewcommand\thepage{\arabic{page}}
}
\newcommand\ps@Preface{
  \pagenumbering{Roman}
  \renewcommand\@oddhead{\selectlanguage{english}\rmfamily\color{black} Sabanc{\i} University CS Department Instructional Use\hfill \hfill Scheduler}
  \renewcommand\@evenhead{\@oddhead}
  \renewcommand\@oddfoot{\foreignlanguage{english}{\textcolor{black}{\centerline{\thepage{}}}}}
  \renewcommand\@evenfoot{\@oddfoot}
  \renewcommand\thepage{\roman{page}}
}
\newcommand\ps@Appendix{
  \renewcommand\@oddhead{}
  \renewcommand\@evenhead{\@oddhead}
  \renewcommand\@oddfoot{}
  \renewcommand\@evenfoot{\@oddfoot}
  \renewcommand\thepage{\arabic{page}}
}
\makeatother
\pagestyle{Standard}
\setlength\tabcolsep{1mm}
\renewcommand\arraystretch{1.3}
\title{SOFTWARE DESIGN SPECIFICATION}
\author{Serta\c{c} Karahoda}
\date{\currenttime}
\begin{document}
\clearpage\setcounter{page}{1}\pagestyle{Standard}
\thispagestyle{Preface}

\vspace*{1.5in}
{\centering\selectlanguage{english}\bfseries\color{black}
 SOFTWARE DESIGN SPECIFICATION (SDS) FOR 
\par}

\bigskip

{\centering\selectlanguage{english}\bfseries\color{black}
SCHEDULER
\par}


\bigskip


\bigskip


\bigskip


\bigskip

{\centering\selectlanguage{english}\bfseries\color{black}
Version 1.0
\par}

{\centering\selectlanguage{english}\bfseries\color{black}
16 March 2015
\par}


\bigskip


\bigskip

{\centering\selectlanguage{english}\bfseries\color{black}
Prepared for:
\par}

{\centering\selectlanguage{english}\bfseries\color{black}
Ramin Armanfar <raminarmanfar@sabanciuniv.edu>

\par}


\bigskip


\bigskip

{\centering\selectlanguage{english}\bfseries\color{black}
Prepared by:
\par}

{\centering\selectlanguage{english}\bfseries\color{black}
Berk \c{C}iri\c{s}ci <berkcirisci@sabanciuniv.edu>

Elif Meri\c{c} <elifmeric@sabanciuniv.edu>

Pamir Mundt <pamirmundt@sabanciuniv.edu>

Serta\c{c} Karahoda <skarahoda@sabanciuniv.edu>

Yavuz Selim Emir <yavuzselim@sabanciuniv.edu>
\par}

\clearpage{\centering\selectlanguage{english}\bfseries\color{black}
SDS
\par}

{\centering\selectlanguage{english}\bfseries\color{black}
RECORD OF CHANGES (Change History)
\par}

\begin{flushleft}
\tablehead{}
\begin{supertabular}{|m{0.47685984in}|m{1.2418196in}|m{1.3587599in}|m{0.23375985in}|m{2.0462599in}|m{0.7337598in}|}
\hline
~

\centering \selectlanguage{english}\color{black} Change number &
~

\centering \selectlanguage{english}\color{black} Date completed &
~

\centering \selectlanguage{english}\color{black} Location of change
(e.g., page or figure \#) &
\centering \selectlanguage{english}\bfseries\color{black} A\newline
M\newline
D &
~

\centering \selectlanguage{english}\color{black} Brief description of
change &
~

\centering\arraybslash \selectlanguage{english}\color{black} Author\\\hline
\centering{0}
 &
\centering{9 April 2015}
 &
All
 &
\centering{A}
 &
\raggedright{The document created}
 &
 \vspace{0.05in}
 Serta\c{c}
 \\\hline
 \centering{1}
 &
\centering{12 April 2015}
 &
Section \ref{sec:intro}, \ref{sec:overview}
 &
\centering{M}
 &
 \raggedright{Introduction and System Overview added}
 &
 \vspace{0.05in}
 Berk
 \\\hline
\end{supertabular}
\end{flushleft}
{\selectlanguage{english}\color{black}
A - ADDED \ M - MODIFIED \ D -- DELETED}

\clearpage{\centering\selectlanguage{english}\bfseries\color{black}
[ put program /system name here ]
\par}

{\centering\selectlanguage{english}\bfseries\color{black}
TABLE OF CONTENTS
\par}

{\selectlanguage{english}\bfseries\color{black}
Section\ \ Page}

\setcounter{tocdepth}{9}
\renewcommand\contentsname{}
\tableofcontents

\bigskip

\bigskip
\clearpage\setcounter{page}{1}\pagestyle{Standard}
\section{INTRODUCTION}
\label{sec:intro}

\subsection{IDENTIFICATION}

This document is the software design document of the program called "Scheduler". \ This document is designed for Ramin Armanfar and all developers who are working in the project. \  This a new project effort, so the version under development is Version 1.0.

\subsection{DOCUMENT PURPOSE, SCOPE, AND INTENDED AUDIENCE}

\subsubsection{Document Purpose}

This document will describe the architect of our system and provide a brief summary for the specifications of software design.

\subsubsection{Document Scope}
The scope of this document is to inform our customer about the architect and the design of our system and inform him about the details of our software.

\subsubsection{Intended Audience for Document}

The intended audience of this document originally will be our customer, Ramin Armanfar and the workers of this project but as we released our document on web environment, anyone who are interested with the project can read our document.


\subsection{DEFINITIONS, ACRONYMS, AND ABBREVIATIONS}


\begin{flushleft}
\tablehead{\hline
\centering \selectlanguage{english}\bfseries\color{black} Term or
Acronym &
\centering\arraybslash \selectlanguage{english}\bfseries\color{black}
Definition\\\hline}
\begin{supertabular}{|m{1.3587599in}|m{5.00806in}|}
Architect &
The person, team, or organization responsible for systems
architecture.
\\\hline
Architectural View &
A representation of a whole system from the perspective of a related set
of concerns.
\\\hline
Architecture &
The fundamental organization of a system embodied in its components, their
relationships to each other, and to the environment, and the principles
guiding its design and evolution.
\\\hline
Design View &
A subset of design entity attribute information that is specifically
suited to the needs of a software project activity.
\\\hline
SDS &
Software Design Specification
\\\hline
Software Design Specification &
A representation of a software system created to facilitate analysis,
planning, implementation, and decision making, A blueprint or model of
a software system. The SDS is used as the primary medium for
communicating software design information.
\\\hline
SRS &
Software Requirements
Specification
\\\hline
System &
A collection of components organized to accomplish a specific function or
set of functions.
\\\hline
System Stakeholder &
An individual, team, or organization (or classes thereof) with interests
in, or concerns, relative to, a system.
\\\hline
Bannerweb &
A course registration system for Sabanc{\i} University
\\\hline
Cross platform &
Systems that can run on different operating system like Windows, Linux and Mac OS
\\\hline
Java & 
An object oriented, cross platform programming language
\\\hline
GPL &
GNU General Public License is a free software license, which guarantees end users the freedoms to use, study, share, and modify the software. 
\\\hline
CRN &
Course Reference Number  
\\\hline

\end{supertabular}
\end{flushleft}

\smallskip

\subsection{DOCUMENT OVERVIEW}

\noindent
Section \ref{sec:overview} of this document describes the system and software purpose, scope and intended audience of program. \ 
It describes the functionalities of the program and how it will work while user is using the program. 

\bigskip

\noindent
Section \ref{sec:considerations} of this document describes the considerations of design like assumptions, dependencies, goals, general constraints and priorities. \ It will also describe the issues which should be solved before completing the design.

\bigskip

\noindent
Section \ref{sec:architecture} of this document describes the system and software architecture from several viewpoints, including, chosen system architecture and interface, discussions about other alternative designs. \ It will describe why this architecture selected to design after the discussions about alternative system designs.

\bigskip

\noindent
Section \ref{sec:description} provides detailed design descriptions for every component defined in the architectural view(s). 

\clearpage\pagestyle{Standard}
\section{SYSTEM OVERVIEW}
\label{sec:overview}

\subsection{SYSTEM AND SOFTWARE PURPOSE}

The purpose of the system under development is to make course schedule for Sabanc\i{} University students. \ The main goal is to help students before and during the registration period. \ While the system will be used by Sabanc{\i} University students,
this document is intended to be read and understood by Teaching Assistants and Instructor of Software Engineering course.

\subsection{SYSTEM AND SOFTWARE SCOPE}

This software system will be a course scheduling application for Sabanc\i{} University students. \ The application is based on a cross platform Java form application. \ The project will be developed in 4 months by 5 software engineers. \ There will be no sponsor or any investor. \ It has GPL license, it can be maintained and contributed by volunteer developers.

\subsection{SYSTEM AND SOFTWARE CONTEXT}

This program will be effective for 2 certain operations. \ One of them will be creating a schedule for the student and the other one will be summarizing graduation specifications.
\bigskip

\noindent
For the scheduler part, program will ask the term from the user and pull the courses from Bannerweb. \ After pulling the courses, program can filter some features of the courses if the user wants to be filtered. \ These filters are about time and days of the courses, course types, time conflicts etc.. \ After choosing filters the new course list will be shown to the user and user can easily add courses to or remove courses from the schedule. \ When the user finishes the operations about the schedule, schedule can be saved by the program and user can load it any time (s)he wants.
\bigskip

\noindent
For the graduation specifications, user can see the remaining courses and necessary processes for graduation with the remaining schedules that user created.

\subsection{INTENDED USERS FOR THE SYSTEM AND SOFTWARE}

The intended users of the system will be the Sabanc\i{} University students who are willing to create a schedule for themselves and need a program which helps them to create a schedule by certain specifications.

\clearpage\pagestyle{Standard}
\section{DESIGN CONSIDERATIONS}
\label{sec:considerations}
\subsection{ASSUMPTIONS AND DEPENDENCIES}
\subsection{GENERAL CONSTRAINTS}
\subsection{GOALS AND GENERAL PRIORITIES}

\clearpage\pagestyle{Standard}
\section{SYSTEM ARCHITECTURAL DESIGN}
\label{sec:architecture}

\clearpage\pagestyle{Standard}
\section{DETAILED DESCRIPTION OF COMPONENTS, SUBSYSTEMS AND MODULES}
\label{sec:description}


\iffalse
\clearpage\pagestyle{Standard}
\section{CONSTRAINTS and stakeholder concerns}

{\selectlanguage{english}\itshape\color{black}
This section of the document shall identify environmental or usability
constraints placed upon the development and use of the system and
software, the stakeholders of the system and software, and their
concerns about the system and software, if any.}

\subsection{CONSTRAINTS}
{\selectlanguage{english}\itshape\color{black}
This subsection shall identify and describe in detail the architectural
and usability constraints that are imposed by the physical environment
or system requirements or the user characteristics.}

\subsubsection{Environmental constraints.}
{\selectlanguage{english}\color{black}
[Insert text here.] }

\subsubsection{System requirement constraints.}
{\selectlanguage{english}\color{black}
[Insert text here.]}

\subsubsection{User characteristic constraints.}
{\selectlanguage{english}\color{black}
[Insert text here.]}

\subsection[STAKEHOLDER
CONCERNS]{\selectlanguage{english}\bfseries\color{black} STAKEHOLDER
CONCERNS}
{\selectlanguage{english}\itshape\color{black}
This subsection shall identify all the system and software stakeholders.
Some categories have already been included, add more categories as
needed. Within each category add the list of stakeholders and their
details. For compliance with ISO/IEC 42010:2007 at a minimum the
following concerns shall be identified and described for the system and
software object of this SSDD: appropriateness of the architected
solution for achieving its desired mission, feasibility of
construction, risks of system construction and operation to all
stakeholders, maintainability, deployability, and evolvability. \ Other
stakeholder concerns for the system and software might be: construction
cost, expected lifetime, cost of operation, cost of maintenance, system
safety, data security and privacy, operator and user safety, etc. For
each concern make a reference to the corresponding stakeholder(s) (a
concern might come from more than one stakeholder).}


\bigskip

{\selectlanguage{english}\itshape\color{black}
The following tabular form is preferred, but not required. \ You may
eliminate inappropriate rows and add categories and concerns as
needed.}

\begin{flushleft}
\tablehead{\hline
\multicolumn{4}{|m{7.42956in}|}{\centering
\selectlanguage{english}\bfseries\color{black} Stakeholder x Concern x
Mitigation Table}\\\hline
\multicolumn{1}{|m{1.2198598in}|}{\centering
{\selectlanguage{english}\bfseries\color{black} Stakeholder}\par

\centering \selectlanguage{english}\bfseries\color{black} Concern} &
\centering \selectlanguage{english}\bfseries\color{black} List of
Stakeholders (e.g. Acquirers, Developers, Testers, Maintainers, Users,
Operators, Auditors, Others) &
\centering \selectlanguage{english}\bfseries\color{black} Stated Concern
&
\centering\arraybslash \selectlanguage{english}\bfseries\color{black}
Mitigation Mechanism or Design Criteria Reference Number\\\hline
\multicolumn{1}{|m{1.2198598in}|}{\selectlanguage{english}\color{black}
Appropriateness of the system and software in fulfilling its
mission(s).} &
~
 &
~
 &
~
\\\hline}
\begin{supertabular}{m{1.2198598in}|m{1.4837599in}|m{2.2962599in}|m{2.19346in}|}
 &
~
 &
~
 &
~
\\\hhline{~---}
 &
~
 &
~
 &
~
\\\hhline{~---}
\multicolumn{1}{|m{1.2198598in}|}{\selectlanguage{english}\color{black}
Feasibility of constructing, testing, verifying and deploying the
system and software.} &
~
 &
~
 &
~
\\\hline
 &
~
 &
~
 &
~
\\\hhline{~---}
 &
~
 &
~
 &
~
\\\hhline{~---}
\multicolumn{1}{|m{1.2198598in}|}{\selectlanguage{english}\color{black}
Risks of constructing, deploying, and using the system and software
object of this SSDD.} &
~
 &
~
 &
~
\\\hline
 &
~
 &
~
 &
~
\\\hhline{~---}
 &
~
 &
~
 &
~
\\\hhline{~---}
\multicolumn{1}{|m{1.2198598in}|}{\selectlanguage{english}\color{black}
Concerns with respect to the deployability of the system and software.}
&
~
 &
~
 &
~
\\\hline
 &
~
 &
~
 &
~
\\\hhline{~---}
 &
~
 &
~
 &
~
\\\hhline{~---}
\multicolumn{1}{|m{1.2198598in}|}{\selectlanguage{english}\color{black}
Concerns with respect to the maintainability and evolvability of the
system and software.} &
~
 &
~
 &
~
\\\hline
 &
~
 &
~
 &
~
\\\hhline{~---}
\multicolumn{1}{|m{1.2198598in}|}{\selectlanguage{english}\color{black}
Concerns with respect to the security of the data the system and
software will handle.} &
~
 &
~
 &
~
\\\hline
 &
~
 &
~
 &
~
\\\hhline{~---}
 &
~
 &
~
 &
~
\\\hhline{~---}
\multicolumn{1}{|m{1.2198598in}|}{\selectlanguage{english}\color{black}
Concerns with respect to the safety of the people interacting with the
system and software.} &
~
 &
~
 &
~
\\\hline
 &
~
 &
~
 &
~
\\\hhline{~---}
 &
~
 &
~
 &
~
\\\hhline{~---}
\multicolumn{1}{|m{1.2198598in}|}{\selectlanguage{english}\color{black}
Cost concerns.} &
~
 &
~
 &
~
\\\hline
 &
~
 &
~
 &
~
\\\hhline{~---}
 &
~
 &
~
 &
~
\\\hhline{~---}
\multicolumn{1}{|m{1.2198598in}|}{\selectlanguage{english}\color{black}
[ list concern ]} &
~
 &
~
 &
~
\\\hline
 &
~
 &
~
 &
~
\\\hhline{~---}
 &
~
 &
~
 &
~
\\\hhline{~---}
\multicolumn{1}{|m{1.2198598in}|}{\selectlanguage{english}\color{black}
[ list concern ]} &
~
 &
~
 &
~
\\\hline
 &
~
 &
~
 &
~
\\\hhline{~---}
 &
~
 &
~
 &
~
\\\hhline{~---}
\end{supertabular}
\end{flushleft}
\clearpage\pagestyle{Standard}
\section[SYSTEM AND SOFTWARE
ARCHITECTURE]{\selectlanguage{english}\bfseries\color{black} SYSTEM AND
SOFTWARE ARCHITECTURE}
{\selectlanguage{english}\itshape\color{black}
This section of the document shall describe with detail every detailed
design entity or component of the system as well as the relationship
and interface between them. These architectural entities, when
integrated together as specified within this document, shall implement
all functions performed by the system in response to an input or in
support of an output as described by the System and Software
Requirements Specification (SSRS). \ All architectural entities or
components shall: be uniquely identi[FB01?]able, be well described,
have clear responsibilities, have well specified interfaces, and have
well described interactions with other architectural entities and any
external systems. A system{\textquoteright}s architecture is usually
described by using a set of different views, typically one for the
developer and others for the customer, users, operators, etc. \ All
necessary views at the architectural level (or high-level design) shall
be clearly described in this section. \ In this section we assume that
the reader is familiar with such architectural description languages.
For compliance with ISO/ISEC 42010:2007 each view shall include
at least the following details: identification, system representation
using the corresponding viewpoint, configuration information, languages
and modeling techniques, and references to detailed or further
descriptions of the viewpoint. \ }

\subsection[DEVELOPER{\textquoteright}S ARCHITECTURAL
VIEW]{\selectlanguage{english}\bfseries\color{black}
DEVELOPER{\textquoteright}S ARCHITECTURAL VIEW}
{\selectlanguage{english}\itshape\color{black}
This subsection contains the descriptions of a system and all of its
major components, using the methods, techniques, and languages from the
developer{\textquoteright}s viewpoint. \ Each viewpoint description
includes the viewpoint identification, description, and diagrammatic
representation. }

\subsection[USER{\textquoteright}S ARCHITECTURAL
VIEW]{\selectlanguage{english}\bfseries\color{black}
USER{\textquoteright}S ARCHITECTURAL VIEW}
{\selectlanguage{english}\itshape\color{black}
This subsection contains the descriptions of a system and all of its
major components, using the methods, techniques, and languages from the
user{\textquoteright}s viewpoint. \ Each viewpoint description includes
the viewpoint identification, description, and diagrammatic
representation. }

\subsubsection{User{\textquoteright}s View Identification}
{\selectlanguage{english}\itshape\color{black}
Identify the view, state the purpose of the view, and identify major
components or processes of the architecture.}

{\selectlanguage{english}\color{black}
[Insert text here.]}

\subsubsection{User{\textquoteright}s View Representation and
Description }
{\selectlanguage{english}\itshape\color{black}
Provide a diagram and description of the user{\textquoteright}s view of
the architecture.}

{\selectlanguage{english}\color{black}
[Insert diagram here.]}

\subsection[Developer{\textquoteright}s View
Identification]{\selectlanguage{english}\bfseries\color{black}
Developer{\textquoteright}s View Identification}
{\selectlanguage{english}\itshape\color{black}
Identify the view, state the purpose of the view, and identify major
components or processes of the architecture.}

{\selectlanguage{english}\color{black}
[Insert text here.]}

\subsubsection[Developer{\textquoteright}s View Representation and
Description ]{Developer{\textquoteright}s View Representation and
Description }
{\selectlanguage{english}\itshape\color{black}
Provide a diagram and description of the developer{\textquoteright}s
view of the architecture.}

{\selectlanguage{english}\color{black}
[Insert diagram here.]}

\subsubsection{Developer{\textquoteright}s Architectural Rationale}
{\selectlanguage{english}\itshape\color{black}
For compliance with ISO/IEC 42010:2007 an Architectural
Description (AD) shall provide the rationale that justified the
architect{\textquoteright}s decisions and selected architectures. An AD
shall also provide evidence of the consideration of other alternative
architectures and the rationales for discarding them.}

{\selectlanguage{english}\color{black}
[Insert rationale here.]}

\subsection[\ [ insert name of viewpoint {]} ARCHITECTURAL
VIEW]{\foreignlanguage{english}{\ }\foreignlanguage{english}{[ insert
name of viewpoint ] ARCHITECTURAL VIEW}}
{\selectlanguage{english}\itshape\color{black}
This subsection contains the descriptions of a system and all of its
major components, using the methods, techniques, and languages from
other than the developer{\textquoteright}s or user{\textquoteright}s
viewpoint. \ Each viewpoint description includes the viewpoint
identification, description, and diagrammatic representation. }


\bigskip

{\selectlanguage{english}\itshape\color{black}
Repeat this subsection for each viewpoint identified.}

\subsubsection{[ insert name of viewpoint ]{\textquoteright}s View
Identification}
{\selectlanguage{english}\itshape\color{black}
Identify the view, state the purpose of the view, and identify major
components or processes of the architecture.}

{\selectlanguage{english}\color{black}
[Insert text here.]}

\subsubsection{[ insert name of viewpoint ]{\textquoteright}s View
Representation and Description }
{\selectlanguage{english}\itshape\color{black}
Provide a diagram of the developer{\textquoteright}s view of the
architecture.}

{\selectlanguage{english}\color{black}
[Insert diagram and descriptions here.]}

\subsection[CONSISTENCY OF ARCHITECTURAL
VIEWS]{\selectlanguage{english}\bfseries\color{black} CONSISTENCY OF
ARCHITECTURAL VIEWS}
{\selectlanguage{english}\itshape\color{black}
For compliance with ISO/IEC 42010:2007 an Architectural
Description (AD) shall include a list of all known inconsistencies
between the architectural views and an analysis of consistency across
all the architectural views.}

\subsubsection{Detail of Inconsistencies between Architectural Views}
{\selectlanguage{english}\color{black}
[Insert text and graphics here.]}

\subsubsection{Consistency Analysis and Inconsistency Mitigations}
{\selectlanguage{english}\itshape\color{black}
For each inconsistency identified above, provide solutions or
mitigations that resolve potential conflicts between the stakeholder
viewpoints.}

{\selectlanguage{english}\color{black}
[Insert text or table here.]}

\section{SOFTWARE DETAILED DESIGN}
{\selectlanguage{english}\itshape\color{black}
This section of the document should describe with detail the design of
the software being described in this document. \ The level of detail of
the design entities and their relationship and interfaces shall be
sufficient to enable software implementers to implement an integrate
each of the described components in order to achieve full
implementation of the software being described in this SSDD. This
section shall specify for each design entity the following information:
purpose, processing, data, interfaces, dependencies and relationships,
concept of execution, needed resources, design rationale, information
for reuse, types of errors, and error handling. \ }


\bigskip

{\selectlanguage{english}\itshape\color{black}
The detailed design must correspond to an existing architectural view,
normally the developer{\textquoteright}s view, but unusual
circumstances may call for other detailed design viewpoints. \ If so,
repeat this subsection as needed for those other viewpoints.}

\subsection[\ DEVELOPER{\textquoteright}S VIEWPOINT DETAILED SOFTWARE
DESIGN]{\foreignlanguage{english}{\ }\foreignlanguage{english}{DEVELOPER{\textquoteright}S
VIEWPOINT DETAILED SOFTWARE DESIGN}}
{\selectlanguage{english}\itshape\color{black}
Identify the viewpoint and make reference to the diagram or model
defining the view.}

{\selectlanguage{english}\color{black}
[Insert text, diagram or model here.]}

\subsection[COMPONENT/ENTITY
DICTIONARY]{\selectlanguage{english}\bfseries\color{black}
COMPONENT/ENTITY DICTIONARY}
{\selectlanguage{english}\itshape\color{black}
This subsection shall list and describe all the detailed design entities
and their corresponding attributes. \ Processing and algorithms, data
and data structures,and detailed descriptions need NOT be included
here, as they will be specified in subsequent sections for each
component or entity listed in the table below.}

\begin{flushleft}
\tablehead{}
\begin{supertabular}{|m{1.0462599in}|m{0.9837598in}|m{1.6712599in}|m{1.2962599in}|m{1.2580599in}|}
\hline
\multicolumn{5}{|m{6.57056in}|}{\centering
\selectlanguage{english}\bfseries\color{black} Component/Entity
Dictionary}\\\hline
\centering \selectlanguage{english}\bfseries\color{black} Name &
\centering \selectlanguage{english}\bfseries\color{black} Type/Range &
\centering \selectlanguage{english}\bfseries\color{black}
Purpose/Function &
\centering \selectlanguage{english}\bfseries\color{black} Dependencies &
\centering\arraybslash \selectlanguage{english}\bfseries\color{black}
Subordinates\\\hline
~
 &
~
 &
~
 &
~
 &
~
\\\hline
~
 &
~
 &
~
 &
~
 &
~
\\\hline
~
 &
~
 &
~
 &
~
 &
~
\\\hline
~
 &
~
 &
~
 &
~
 &
~
\\\hline
~
 &
~
 &
~
 &
~
 &
~
\\\hline
\end{supertabular}
\end{flushleft}
\subsection[COMPONENT/ENTITY DETAILED
DESIGN]{\selectlanguage{english}\bfseries\color{black} COMPONENT/ENTITY
DETAILED DESIGN}
\subsubsection{Detailed Design for Component/Entity: [ insert
Component/Entity name here ]}
\paragraph[\ Introduction/Purpose of this
Component/Entity]{\ Introduction/Purpose of this Component/Entity}
{\selectlanguage{english}\color{black}
[ insert your text here ]}

\paragraph[Input for this Component/Entity]{Input for this
Component/Entity}
{\selectlanguage{english}\color{black}
[ insert your text here ]}

\paragraph{Output for this Component/Entity}
{\selectlanguage{english}\color{black}
[ insert your text here ]}

\paragraph{Component/Entity Process to Convert Input to Output}
{\selectlanguage{english}\color{black}
[ insert your text here ]}

\paragraph{Design constraints and performance requirements of this
Component/Entity}
{\selectlanguage{english}\color{black}
[ insert your text here ]}

\subsubsection{Detailed Design for Component/Entity: [ insert
Component/Entity name here ]}
\paragraph[\ Introduction/Purpose of this
Component/Entity]{\ Introduction/Purpose of this Component/Entity}
{\selectlanguage{english}\color{black}
[ insert your text here ]}

\paragraph{Input for this Component/Entity}
{\selectlanguage{english}\color{black}
[ insert your text here ]}

\paragraph{Output for this Component/Entity}
{\selectlanguage{english}\color{black}
[ insert your text here ]}

\paragraph{Component/Entity Process to Convert Input to Output}
{\selectlanguage{english}\color{black}
[ insert your text here ]}

\paragraph{Design constraints and performance requirements of this
Component/Entity}
{\selectlanguage{english}\color{black}
[ insert your text here ]}

\subsubsection{Detailed Design for Component/Entity: [ insert
Component/Entity name here ]}
\paragraph[\ Introduction/Purpose of this
Component/Entity]{\ Introduction/Purpose of this Component/Entity}
{\selectlanguage{english}\color{black}
[ insert your text here ]}

\paragraph{Input for this Component/Entity}
{\selectlanguage{english}\color{black}
[ insert your text here ]}

\paragraph{Output for this Component/Entity}
{\selectlanguage{english}\color{black}
[ insert your text here ]}

\paragraph{Component/Entity Process to Convert Input to Output}
{\selectlanguage{english}\color{black}
[ insert your text here ]}

\paragraph{Design constraints and performance requirements of this
Component/Entity}
{\selectlanguage{english}\color{black}
[ insert your text here ]}

\subparagraph[{\dots}]{{\dots}}
\subsubsection{Detailed Design for Component/Entity: [ insert
Component/Entity name here ]}
\paragraph[\ Introduction/Purpose of this
Component/Entity]{\ Introduction/Purpose of this Component/Entity}
{\selectlanguage{english}\color{black}
[ insert your text here ]}

\paragraph{Input for this Component/Entity}
{\selectlanguage{english}\color{black}
[ insert your text here ]}

\paragraph{Output for this Component/Entity}
{\selectlanguage{english}\color{black}
[ insert your text here ]}

\paragraph{Component/Entity Process to Convert Input to Output}
{\selectlanguage{english}\color{black}
[ insert your text here ]}

\paragraph{Design constraints and performance requirements of this
Component/Entity}
{\selectlanguage{english}\color{black}
[ insert your text here ]}

\subsection{DATA DICTIONARY}
{\selectlanguage{english}\itshape\color{black}
This subsection shall list and describe all the data and data structures
defined and/or used by the components and entities specified above.
\ For each data item or structure indicate where it is defined,
referenced, and modified.}

\begin{flushleft}
\tablehead{}
\begin{supertabular}{|m{0.9837598in}|m{0.9212598in}|m{1.8587599in}|m{1.2962599in}|m{1.1330599in}|}
\hline
\multicolumn{5}{|m{6.50806in}|}{\centering
\selectlanguage{english}\bfseries\color{black} Data Dictionary}\\\hline
\centering \selectlanguage{english}\bfseries\color{black} Name &
\centering \selectlanguage{english}\bfseries\color{black} Type/Range &
\centering \selectlanguage{english}\bfseries\color{black} Defined
by{\dots} &
\centering \selectlanguage{english}\bfseries\color{black} Referenced
by{\dots} &
\centering\arraybslash \selectlanguage{english}\bfseries\color{black}
Modified by{\dots}\\\hline
~
 &
~
 &
~
 &
~
 &
~
\\\hline
~
 &
~
 &
~
 &
~
 &
~
\\\hline
~
 &
~
 &
~
 &
~
 &
~
\\\hline
~
 &
~
 &
~
 &
~
 &
~
\\\hline
~
 &
~
 &
~
 &
~
 &
~
\\\hline
\end{supertabular}
\end{flushleft}

\bigskip


\bigskip

\clearpage\setcounter{page}{1}\pagestyle{Standard}
\section[APPENDIX A. \ [insert name
here{]}]{\selectlanguage{english}\bfseries\color{black} APPENDIX A.
\ [insert name here]}
{\selectlanguage{english}\itshape\color{black}
Include copies of specifications, mockups, prototypes, etc. supplied or
derived from the customer. \ Appendices are labeled A, B, {\dots}n.
\ \ Reference each appendix as appropriate in the text of the document.
}

{\selectlanguage{english}\color{black}
\ [ insert appendix A here ]}

\clearpage\setcounter{page}{1}\pagestyle{Standard}
\section[APPENDIX B. \ [insert name
here{]}]{\selectlanguage{english}\bfseries\color{black} APPENDIX B.
\ [insert name here]}

\bigskip

{\selectlanguage{english}\color{black}
[ insert appendix B here ]}


\bigskip

\fi
\end{document}
